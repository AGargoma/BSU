\documentclass[12pt, a4paper]{article}
\usepackage[a4paper, top=2cm, bottom=2cm, left=3cm, right=1cm]{geometry}
\usepackage[T2A]{fontenc}
\usepackage[utf8]{inputenc}
\usepackage[russian]{babel}
\usepackage{listingsutf8}
\usepackage{float}
\usepackage{graphicx}
\usepackage{mathtools}
\usepackage{amsmath}
\usepackage{xcolor}

\graphicspath{{images/}}

\definecolor{codegreen}{rgb}{0,0.6,0}
\definecolor{codegray}{rgb}{0.5,0.5,0.5}
\definecolor{codepurple}{rgb}{0.58,0,0.82}
\definecolor{backcolour}{rgb}{0.95,0.95,0.92}


\lstdefinestyle{mystyle}{
	backgroundcolor=\color{backcolour},   
	commentstyle=\color{codegreen},
	keywordstyle=\color{magenta},
	numberstyle=\tiny\color{codegray},
	stringstyle=\color{codepurple},
	basicstyle=\ttfamily\footnotesize,
	breakatwhitespace=false,         
	breaklines=true,                 
	captionpos=b,                    
	keepspaces=true,                 
	numbers=left,                    
	numbersep=5pt,                  
	showspaces=false,                
	showstringspaces=false,
	showtabs=false,                  
	tabsize=4
}

\lstset{inputencoding=utf8/koi8-r, style=mystyle}
\begin{document}
	
	\begin{titlepage}
		\centering{
			\MakeUppercase{\textbf{БЕЛОРУССКИЙ ГОСУДАРСТВЕННЫЙ УНИВЕРСИТЕТ}} \\[0.4cm]
			
			Факультет прикладной математики и информатики \\[0.4cm]
			
			\vspace{15em}
			
			{\large\bfseries{Отчёт по лабораторной работе №3}}
			
			{\large\bfseries{<<Интерполяционный кубический сплайн>>}} \\[4cm]
			
			\noindent
			\begin{tabular}{p{0.6\textwidth}p{0.6\textwidth}}
				& Выполнил: \\
				& Гаргома А.\,О. \\[1cm]
				& Преподаватель: \\
				& Горбачева Ю. Н.
				
			\end{tabular}
			
			\vfill
			
			{\normalsize Минск, 2020}
			
		}
	\end{titlepage}
	
	\tableofcontents
	
	\section{Постановка задачи}
	
	Вычислить значения заданной функции $f(x)$ в узлах интерполяции 
	$x_i = a+ih, i = \overline{0,n}$ на отрезке $[a,b]$ с шагом $h = \frac{b-a}{n}$.
	По полученной таблице ${x_,, f(x_i)}$ построить интерполяционный кубический сплайн 
	$S_3(x)$ с дополнительными условиями:
	
	\begin{equation}
	S_3'(a) = f'(a),\ S_3'(b) = f'(b)
	\label{eq:condition_1}
	\end{equation}
	
	\begin{equation}
	S_3''(a) = f''(a),\ S_3''(b) = f''(b)
	\label{eq:condition_2}
	\end{equation}
	
	\begin{equation}
	S_3''(a)=0,\ S_3''(b) = 0
	\label{eq:condition_3}
	\end{equation}
	
	Убедиться, что значения функций у узлах интерполяции совпадают со значениями 
	сплайна для всех типов дополнительных условий. В точках
	$\overline{x_j} = a+(j+0.5)h$, $j=\overline{0, n-1}$ вычислить значения сплайна
	для всех типов дополнительных условий и сравнить со значениями функции $f(x)$ в
	этих точках, т.е. найти 
	
	\begin{equation}
	\max_{j=0,\ n-1}|f(\overline{x_j}) - S_3(\overline{x_j})|
	\label{eq:criteria}
	\end{equation}
	
	В одной системе координат построить график функции $f(x)$  и графики кубического 
	сплайна для трёх типов дополнительных условий.
	
	Функция $f(x) = \cos(x^2+x)$, $ n = 20, [a,b]=[-2,2]$ 
	
	\section{Теория}
	
	Кубический сплайн имеет вид:
	
	\begin{equation}
	P_{i,3}(x) = a_i + \beta_i(x-x_{i-1}) + \frac{\gamma_i(x-x_{i-1})^2}{2} + \frac{\delta_i(x-x_{i-1})^3}{6}
	\end{equation}
	
	Коэффициенты $\alpha_i,\beta_i,\ \gamma_i,\ \delta_i$ находится из соотношений
	
	\begin{flalign}
	&\alpha_i = f(x_{i}) \\
	&\beta_i = \frac{f(x_i)-f(x_{i-1})}{h_i}+ \frac{(2*\gamma_i+\gamma_{i-1})*h_i}{6}\\
	&\delta_i = \frac{\gamma_i-\gamma_{i-1}}{h_i}\\
	&c_i*\gamma_{i-1}+2*\gamma_i + e_i*\gamma_{i+1} = b_i\\
	&c_i = \frac{h_i}{h_i+h_{i+1}}\\
	&e_i = \frac{h_{i+1}}{h_i+h_{i+1}}\\
	&b_i = 6*f[x_{i-1},x_i,x_{i+1}] \\
	&h_i = x_i - x_{i-1}, i=\overline{1, n-1}\\
	\end{flalign}
	
	\begin{enumerate}
		\item $S_3'(a) = f'(a),\ S_3'(b) = f'(b)$
		
		\begin{align}
		& 2*\gamma_0 + \gamma_1 = 6*f[x_0,x_1,x_2]\\
		& \gamma_{n-1} + 2*\gamma_n = 6*f[x_{n-1},x_n,x_n]
		\end{align}
		
		\item $S_3''(a) = f''(a),\ S_3''(b)=f''(b)$
		
		\begin{equation}
		\gamma_0 = \frac{f''(a)}{2},\ \gamma_{n} = \frac{f''(b)}{2}
		\end{equation}
		
		\item $S_3''(a) = 0,\ S_3''(b) = 0$
		
		\begin{equation}
		\gamma_0 = \gamma_{n} = 0
		\end{equation}
		
	\end{enumerate}
	
	\section{Программа}
	
	\lstinputlisting[language=python]{SplineInterpolation.py}
	
	\section{Результаты}
	
	\begin{figure}[H]
		\centering
		\includegraphics[width=\linewidth]{1}
		\caption{Первое дополнительное условие}
		\label{fig:case_1}
	\end{figure}
	
	
	\begin{figure}[H]
		\centering
		\includegraphics[width=\linewidth]{2}
		\caption{Второе дополнительное условие}
		\label{fig:case_2}
	\end{figure}
	
	
	
	\begin{figure}[H]
		\centering
		\includegraphics[width=\linewidth]{3}
		\caption{Третье дополнительное условие}
		\label{fig:case_3}
	\end{figure}
	
	
	
	\begin{table}[H]
		\centering
		\begin{tabular}{|l|l|}
			\hline
			Сплайн	& $\max_{j=0,\ n-1}|f(\overline{x_j}) - S_3(\overline{x_j})|$ \\ \hline
			1   	&     0.0019595685331641882	\\ \hline
			2   	&     0.003589376437711289		\\ \hline
			3   	&     0.04649748434138057		\\ \hline
		\end{tabular}
	\end{table}
	
	\section{Выводы}
	
	Из таблицы погрешностей выше видно, что естественные граничные условия имеют большую погрешность, нежели другие граничные условия.
	
\end{document}
